\documentclass{instructions}

\usepackage{xspace}
\newcommand{\eg}{\textit{e.g.}\xspace}
\newcommand{\etal}{\textit{et al.}\xspace}
\newcommand{\ie}{\textit{i.e.}\xspace}
\newcommand{\etc}{\textit{etc.}\xspace}
\newcommand{\vs}{\textit{vs.}\xspace}

\newcommand\bs{\char`\\}

%\week{1}
\title{ROS: First Encounter}
\date{8 December 2015}

\summary{
}

\objectives{
}

\challenges{
}

\begin{document}

\maketitle

\intro

Normally, ROS should be installed on your system. Test it quickly by openning
two terminals. In the first one, run \sh{roscore} and in the second one, run
\sh{rosrun rviz rviz}.

This should open RViz, the 3D visualization tool that comes with ROS.

\more{If this does not work, come and see me!}

For today's tutorial, we will all use the same \sh{roscore} (\ie we will all be
part of the same ROS network). So:

\begin{itemize}
    \item Kill the \sh{roscore} process you've just launched,
    \item Type: \sh{export ROS_MASTER_URI=http://<ip on the whiteboard>:11311}
\end{itemize}

\note{
    You must set \sh{ROS_MASTER_URI} in every terminal used to start a ROS
    application! Do not forget to type it when you open a new terminal.
    Alternatively, you can add it to your \sh{.bashrc} file to automatically set
    it for every terminal.
}

\part{On the way to the treasure}

\note{
    As you will discover, this document does not provide much details on how to
    complete the below 'Goals': try, experiment, find code examples online (in
    particular here: \url{http://wiki.ros.org/ROS/Tutorials}), ask your
    neighbour, and blame my laziness!
}

\step{Find your robot}

Now that everything is setup on your system, find out the colour of your robot!

Explore and test the core ROS tools (\sh{rosnode}, \sh{rostopic},
\sh{rosservice}, etc.), and find a way to 'give a hug' to your robot.

\more{
No need to write any code for that yet!
}

\step{Move your robot out of the crowd}

Can you get your robot to move?

\more{
Still no need to write code!
}

While you are at it, print out the position of your robot


\step{A keyboard controller}

Some code, at last!

Fire up your favorite editor, and write a ROS Python node that uses the keys
WASD to move the robot accordingly.

%\begin{pythoncode}
%    TODO: add skeleton to read keys
%\end{pythoncode}

\part{Where is my gold?}

\step{Localize the treasure}

Launch \sh{rviz} and find a way to display the position (\ie the TF frame) of
the treasure.

From the command line, try to obtain the exact location of the treasure, using
\sh{tf_echo} (\sh{rosrun tf tf_echo}).

\step{Orient your robot}

Your robot only output its 6D pose as a \sh{geometry_msgs/PoseStamped} message
(use \sh{rostopic show} and \sh{rostopic echo} if necessary).

Write a small Python node \sh{robot_frame.py} that reads this pose every time the robot publishes it,
and broadcasts it back as a TF frame (the TF tutorials
\url{http://wiki.ros.org/tf/Tutorials} have useful examples!)

\step{Write a first autonomous controller}

Alright, time to write a first \emph{real} ROS package.

To make things more interesting, we are going to write a new node in C++.

First, create a package skeleton with \sh{catkin_create_pkg}, build it, and install
it:

\begin{shcode}
$ catkin_create_pkg my_robot_controller tf roscpp geometry_msgs
$ cd my_robot_controller
$ mkdir build && cd build
$ ccmake ..
\end{shcode}

\more{If \sh{ccmake} is not installed on your system, it is the right time to
    install it: \sh{sudo apt-get install cmake-curses-gui}
}

Make sure you set \sh{CMAKE_BUILD_TYPE} to \sh{Release} and
\sh{CMAKE_INSTALL_PREFIX} to your ROS install prefix (for instance
\sh{/opt/ros/jade}) or, yet better, a dedicated development prefix like
\sh{/home/<username>/dev}. Press {\tt c} to configure, then {\tt g} to generate
the makefiles. Quit \sh{ccmake} and then:

\begin{shcode}
$ make
$ sudo make install
\end{shcode}

If everything went fine, CMake should have installed an handful of files, but
nothing has been compiled yet ...well, that's fair since we did not write any
code yet.

Create a new file \sh{src/my_robot_controller_node.cpp} and copy-paste the
following code:

\begin{cppcode}
#include <ros/ros.h>
#include <tf/transform_listener.h>
#include <geometry_msgs/Twist.h>

int main(int argc, char** argv){
  ros::init(argc, argv, "my_robot_controller");

  ros::NodeHandle node;

  ros::Publisher cmd_vel = node.advertise<geometry_msgs::Twist>("/<your robot>/cmd_vel", 10);

  tf::TransformListener listener;

  ros::Rate rate(10.0);
  ROS_INFO("My Preeeeecious!");

  while (node.ok()){
    tf::StampedTransform transform;
    try{
      listener.lookupTransform("/<your robot frame>", "/map",  
                               ros::Time(0), transform);
    }
    catch (tf::TransformException ex){
      ROS_ERROR("%s",ex.what());
      ros::Duration(1.0).sleep();
    }

    geometry_msgs::Twist twist;
    twist.angular.z = 1.;
    twist.linear.x = 0.5;
    cmd_vel.publish(twist);

    rate.sleep();
  }
  return 0;
};
\end{cppcode}

Modify \sh{CMakeLists.txt} to add your newly created node to the compilation (uncomment
the lines 131, 135, 138-140, 157-161), build the project, install it and run it:

\begin{shcode}
$ cd build
$ make
$ sudo make install
$ rosrun my_robot_controller my_robot_controller_node
\end{shcode}

\note{
You also need to have your TF broadcaster running, otherwise TF will complaint
that it could not find your robot's frame.
}

Modify the code to:
\begin{itemize}
    \item Print (using \sh{ROS_INFO} or \sh{ROS_INFO_STREAM}) the distance and
        angle to the treasure,
    \item Adapt the angular velocity to actually move toward the treasure
\end{itemize}


The first to reach the chest gets... well, the treasure! Ask me!

\step{Simplify the launch procedure}

Instead of having to manually launch the TF broadcaster and then, the robot
controller, we can create a \emph{launch file} that does that for us.

First, move your TF broadcaster node to the package \sh{my_robot_controller} (by convention, in the
\sh{nodes/} directory) and modify \sh{CMakeLists.txt} to install it along the
node \sh{my_robot_controller_node}.

Check it runs fine by calling:

\begin{shcode}
$ rosrun my_robot_controller robot_frame.py
\end{shcode}

Now, create a launch file \sh{control.launch} in \sh{launch/}:

\begin{xmlcode}
<launch>
    <node pkg="my_robot_controller" type="robot_frame.py" name="broadcaster" output="screen" />
    <node pkg="my_robot_controller" type="my_robot_controller_node" name="controller" output="screen" />
</launch>
\end{xmlcode}

Update \sh{CMakeLists.txt} to install the launch file, and test it:

\begin{shcode}
$ roslaunch my_robot_controller control.launch
\end{shcode}


\part{What next?}

Depending on your interests, here a selection of interesting ROS packages/examples
(sorted by increasing complexity) that I encourage you
to explore to gain more experience with ROS:

\begin{itemize}
    \item 6D face tracking with a webcam\\ \url{https://github.com/severin-lemaignan/attention-tracker}
    \item Accessing the Kinect point-cloud\\\url{http://wiki.ros.org/freenect_camera}
    \item ROS with Nao\\\url{http://wiki.ros.org/nao/}
    \item 2D map creation, SLAM and path planning (requires MORSE) \\
        \url{http://www.openrobots.org/morse/doc/latest/user/advanced_tutorials/ros_nav_tutorial.html}
    \item 3D object recognition (much more difficult!)\\\url{http://wg-perception.github.io/ork_tutorials/index.html}
\end{itemize}

\end{document}
